% Central LaTeX style settings for the PDF output (structure, fonts, colors, headings, layout, lists)

%-------------------------------------------
% BIBLIOGRAPHY
%-------------------------------------------

% The bibliography font is handled differently:
% Set bibliography to footnote size that is sufficient as the bibliography is not read from top to bottom, but serves just as a reference in rare cases; comment out if you do not have a lot of references and can spare some space
\renewcommand{\bibfont}{\footnotesize}


% You can use ``same'' (same font as your document's), ``sf'', ``tt''  or ``rm''.
% Also see https://www.ctan.org/pkg/url
\urlstyle{tt}


%-------------------------------------------
% PDF SETUP
%-------------------------------------------

% Set up page geometry
\usepackage[
    paperwidth=\dimexpr\bookwidth+\bleed\relax,
    paperheight=\dimexpr\bookheight+\bleed+\bleed\relax,
    inner=\margininside,
    outer=\dimexpr\marginoutside+\bleed\relax,
    top=\dimexpr\margintop+\bleed\relax,
    bottom=\dimexpr\marginbottom+\bleed\relax,
    includehead,
    includefoot,
    headheight=24pt
]{geometry}
%-------------------------------------------
% TYPOGRAPHY
%-------------------------------------------

% For custom fonts for certain parts of the book, use addtokomafont:

\addtokomafont{subsection}{\large\sffamily\bfseries} % For subsection headings.
\addtokomafont{section}{\LARGE\sffamily\bfseries} % For section headings.

\addtokomafont{chapter}{\Huge\sffamily\bfseries}
\addtokomafont{chapterprefix}{\Huge\sffamily\bfseries}
\addtokomafont{caption}{\sffamily} % For figure and table captions
\addtokomafont{captionlabel}{\sffamily} % For labels in captions

%-------------------------------------------
% HEADINGS
%-------------------------------------------

% Adding a rule below the chapter title
\renewcommand*{\chapterheadendvskip}{%
  \vspace{.5\baselineskip}%
  \hrule height 2pt
  \vspace{\baselineskip}%
}


%-------------------------------------------
% HEADERS AND FOOTERS
%-------------------------------------------

% Package to control the headers and footers
\usepackage{scrlayer-scrpage}
\automark[chapter]{chapter}
% Clear all existing header/footer settings
\clearpairofpagestyles

% Define the footer content

% Places the page number in the outer footer (right side on odd pages, left side on even pages)
\ofoot*{\pagemark}
% Clears the inner footer, leaving it empty
\ifoot*{}


% Define the header content

% Places the section/chapter heading in the outer header (right side on odd pages, left side on even pages)
\ohead*{\headmark}


 % Add a 0.4pt line below the header
\KOMAoptions{headsepline=0.4pt}

% Redefine the plain page style to prevent headers on chapter and part pages
\renewcommand{\chapterpagestyle}{empty}
\renewcommand{\partpagestyle}{empty}


%-------------------------------------------
% VERTICAL SPACING
%-------------------------------------------

% Set a minimal space between each item of itemized and enumerated lists
%\setitemize{nosep}
%\setenumerate{nosep}
%\setdescription{nosep}

%-------------------------------------------
% TABLE OF CONTENTS
%-------------------------------------------

% Define the indentation in the table of contents

\RedeclareSectionCommand[tocpagenumberbox=\mbox]{part}
\RedeclareSectionCommand[tocpagenumberbox=\mbox,tocnumwidth=2em]{chapter}
\RedeclareSectionCommand[tocpagenumberbox=\mbox,tocindent=1.5em,tocnumwidth=4.5em]{section}
\RedeclareSectionCommand[tocpagenumberbox=\mbox,tocindent=1.5em,tocnumwidth=4.5em]{subsection}

\setcounter{tocdepth}{1}  % Less detailed TOC

% Uncomment to fix overfull boxes quickly (at the cost of quality)
% \emergencystretch=3em


%-------------------------------------------
% LATEX TYPE 1/OPENTYPE FONTS SELECTION
%-------------------------------------------
% See https://www.overleaf.com/learn/latex/Font_typefaces and https://www.tug.org/FontCatalogue for more LaTeX font faces

\ifluatex
    \setmainfont{Crimson Pro}
    \setsansfont[Scale=MatchLowercase]{Source Sans Pro}
    \setmonofont[Scale=MatchLowercase]{Source Code Pro}
    \setmathfont{XITS Math}
    \gdef\fontlicensetext{This document uses Crimson Pro, Source Sans Pro, Source Code Pro, and XITS Math fonts, which are licensed under the SIL Open Font License, Version 1.1.}
\else
    \usepackage{libertinus}
    \usepackage[scale=1.02]{sourcecodepro}
    \gdef\fontlicensetext{This document uses Linux Libertine and Source Code Pro fonts, which are licensed under the SIL Open Font License, Version 1.1.}
\fi


\newcommand{\definition}[2]{%
    \begin{tcolorbox}[colframe=black, leftrule=4pt, left=5pt, right=5pt, top=5pt, bottom=5pt, arc=1pt]
    \textbf{\textsf{#1}} \,$\bullet$\, #2
    \end{tcolorbox}%
}

\graphicspath{{images/}}
